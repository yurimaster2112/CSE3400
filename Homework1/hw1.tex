\documentclass[12pt]{article}
\usepackage{amsmath,amssymb,epsfig,boxedminipage,helvet,theorem,endnotes,version}
\usepackage{mathtools}
\newcommand{\remove}[1]{}
\setlength{\oddsidemargin}{-.2in}
\setlength{\evensidemargin}{-.2in}
\setlength{\textwidth}{6.5in}
\setlength{\topmargin}{-0.8in}
\setlength{\textheight}{9.5in}


\newtheorem{theorem}{Theorem} 
\newtheorem{definition}{Definition}

\newcommand{\gen}{\mbox{\tt Gen}}
\newcommand{\enc}{\mbox{\tt Enc}}
\newcommand{\dec}{\mbox{\tt Dec}}

\newcommand{\sol}{{\bf Solution}: }

\newcommand{\zo}{\{0,1\}}
\newcommand{\zoell}{\{0,1\}^\ell}
\newcommand{\zon}{\{0,1\}^n}
\newcommand{\zok}{\{0,1\}^k}

%% eg \priv{eav}{n,b} for def 3.9 style, or \priv{eav}{n} for def 3.8
\newcommand{\priv}[2]{${\tt Priv}_{A,\Pi}^{\tt #1}(#2)$}

\newcommand*\concat{\mathbin{\|}}

\newcommand\xor{\oplus}

\newcommand{\handout}[2]{
\renewcommand{\thepage}{\footnotesize CSE 3400/CSE 5850, #1, p. \arabic{page}}
\begin{center}

\noindent
{\bf UConn, CSE Dept.}

\noindent
{\bf Spring 2023}

\noindent
{\bf CSE 3400/CSE 5080: Introduction to Computer and Network Security \\(or Introduction to Cybersecurity)}
\end{center}

\begin{center}
{\Large #1}
\end{center}
}



\begin{document}

\handout{Assignment 1}{}

\noindent
{Instructor: Prof.~Ghada Almashaqbeh}

\noindent
{Posted: 1/26/2023}

\noindent
{Submission deadline: 2/3/2023, 11:59 pm} \\\\


\noindent{\bf Problem 1 [30 points]\\}
Write a program that increments a counter $2^{10}, 2^{20}$, and $2^{30}$ times (so the initial counter value will be 0, then 1, 2, ..., up to $2^{10}$, then repeat with 0, 1, ...., $2^{20}$, and so on). 
\begin{enumerate}
\item For each counter value, in each iteration compute the bitwise XOR of the counter with the value 8003 (so let the counter be $ctr = 0$, compute $ctr \oplus 8003$, then set $ctr = 1$ and compute the XOR, and so on). Do that for $2^{10}, 2^{20}$, and $2^{30}$ counter values. Measure how many seconds your program takes to run in each case (list the machine specifications and the programming language you have used). Estimate how many years your program would take to do this operation when the counter is incremented $2^{330}$ times.

\item Repeat the previous counter increments but now in each iteration compute the division of the counter value by the number 4009. Measure how many seconds your program takes to run in each case (list the machine specifications and the programming language you have used). Estimate how many years your program would take to do this operation when the counter is incremented $2^{330}$ times. 
\end{enumerate}


% ***********************************************************************************************

\noindent{\bf Problem 2 [40 points]\\}
Let $G_0:\zo^n \to \zo^{2n}$ and $G_1:\zo^n \to \zo^{2n}$ be PRGs. For each of the following constructions, prove whether it is a PRG or not. To prove insecurity you have to show an attack against the scheme and provide an informal argument about the attack success probability. For security provide a convincing informal argument why it is a PRG.
\begin{enumerate}
\item $G_2(s \concat t \concat z)= G_1(s) \concat (t \xor z)$ 

\item $G_3(s)= G_0(\bar{s}) \oplus 1^{2n}$

\item $G_4(s \concat z) =  ((s \concat z) \oplus G_0(s)) \concat G_0(s)$

\item $G_5(s) = \mathsf{msb}(G_0(s)) \concat G_0(s) \concat G_1(z)$, where $\mathsf{msb}$ is the most significant bit.
\end{enumerate} 
($\concat$ is the concatenation operation, $\oplus$ is the bitwise XOR operation, $s$ and $z$ are random strings each of which is of length $n$ bits, and $t$ is any string---not necessarily random---of length $n$ bits.)\\


% ***********************************************************************************************
\newpage
\noindent{\bf Problem 3 [30 points]\\}
\noindent\emph{Part 1:} Answer the following for the monoalphabetic substitution cipher.
\begin{enumerate}
\item Assume you have a CPA attacker, how much plaintext this attacker must request encrypting in order to recover the full key (which is the permutation of the letters)? why? 

\item Will your answer change for the previous question if we consider  a CCA attacker? why?

\item If we use this cipher with totally random messages, will the letter frequency attack that we studied in class apply here as well? why?
\end{enumerate}


\noindent\emph{Part 2:} Answer the following about OTP (just provide informal convincing arguments, no need for formal proofs).
\begin{enumerate}
\item Assume that Alice reuses the pad for every 500 messages she sends to Sponge. Is this a secure way to protect the confidentiality of her communication with Sponge given that they will exchange only 100 messages? why?

\item Is the original OTP, as we took it in class, secure if used to encrypt messages each of which is of length 1 bit? why?
\end{enumerate}


% ***********************************************************************************************

\end{document} 